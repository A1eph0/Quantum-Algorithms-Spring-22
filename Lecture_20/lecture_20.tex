\documentclass[11.5pt, paper=a4]{article}

\usepackage[utf8]{inputenc}
\usepackage[english]{babel}
\usepackage[T1]{fontenc}

\usepackage{amsmath, amssymb, amscd, amsthm, amsfonts, mathtools}
\usepackage[left=2cm, right=2cm, top=1.5cm]{geometry}

\usepackage{graphicx}
\usepackage{hyperref}
\usepackage{physics}
\usepackage{tikz}
\usepackage{url}
\usepackage[square,numbers]{natbib} \usepackage{tabularx}
\usepackage{quantikz}

\usepackage{braket}
\usepackage{thmtools}
\usepackage{float}
\usepackage{mdframed}

%%% Theorem Style
\theoremstyle{definition}
\newtheorem{theorem}{Theorem}[section]
\newtheorem{definition}[theorem]{Definition}
\newtheorem{lemma}[theorem]{Lemma}
\newtheorem{conjecture}[theorem]{Conjecture}
\newtheorem{corollary}[theorem]{Corollary}
\newtheorem{claim}{Claim}
\newmdtheoremenv{boxthm}{Theorem}

\numberwithin{theorem}{section}

%% Autoref prefixes
\renewcommand{\sectionautorefname}{Section}
\renewcommand{\subsectionautorefname}{Section}
\renewcommand{\subsubsectionautorefname}{Section}
\renewcommand{\figureautorefname}{Figure}
\def\theoremautorefname{Theorem}
\def\lemmaautorefname{Lemma}
\def\definitionautorefname{Definition}
\def\conjectureautorefname{Conjecture}
\def\algorithmautorefname{Algorithm}

%% Writing algorithms

\usepackage{algorithm} % captioning 
\usepackage{algpseudocode}

% \def\NoNumber#1{{\def\alglinenumber##1{}\State #1}\addtocounter{ALG@line}{-1}}

\title{Quantum Algorithms, Spring 2022: Lecture 20 Scribe}

\author{Nithish Raja, Shodasakshari Vidya}

\date{\today}

\begin{document}

\maketitle

\section{Recap}

\noindent The grover iterate used in grover;s search is as follows
\begin{equation*}
    G = \begin{bmatrix}
    \cos\theta & -\sin\theta\\
    \sin\theta & -\cos\theta\\
    \end{bmatrix} \tag{$\frac{\theta}{2} = \arcsin{\sqrt{\frac{M}{N}}}$}
\end{equation*}

\noindent Eigenvalues and eigenvectors of $G$ are the following
\begin{equation*}
\begin{split}
    \ket{\lambda_{\pm}} & = e^{\pm i\theta}\\
    \ket{\lambda_{\pm}} & = \frac{\ket{\omega}\pm i\ket{S_{\bar{\omega}}}}{\sqrt{2}}\\
\end{split}
\end{equation*}

\noindent To get an estimate of $M$, we apply QPE with the grover iterate and estimate $\theta$. Using the estimated $\Tilde{\theta}$, we obtain an estimate $\Tilde{M} = N\sin^2\frac{\Tilde{\theta}}{2}$.\\ \\
\noindent However, to use QPE, we require an eigenvector of $G$ or superposition of eigenvectors. Since $H^{\otimes n}\ket{0}^{\otimes n} = \ket{s} = \sqrt{\frac{N-M}{N}}\ket{S_{\bar{\omega}}} + \sqrt{\frac{M}{N}}\ket{\omega}$, we use that instead.\\ \\

\noindent By setting QPE accuracy to $|\delta| = \frac{1}{\sqrt{N}}$, we can estimate $\Tilde{M} = M(1\pm O(\frac{1}{\sqrt{M}}))$.\\ \\

\noindent Setting $t = \frac{n}{2} + \lceil\log(\frac{1}{\epsilon})\rceil$, we get a success probability $\ge 1 - \epsilon$. We can get a constant success probability in running time $O(\sqrt{N})$.

\section{Optimality of quantum search}

Consider a setting where only one solution ($\omega$) exists among $N$ and $\omega$ can be any of the $N$ possible values.\\
Let two circuits be defined as follows
\begin{equation*}
    \ket{\psi^\omega_T} = U_TU_\omega U_{T-1}U_\omega\hdots U_1U_\omega\ket{s} \tag{$U_\omega = I - 2\ket{\omega}\bra{\omega}$}
\end{equation*}
\begin{equation*}
    \ket{\psi_T} = U_TU_{T-1}\hdots U_1\ket{s} \tag{$\ket{s} = H^{\otimes n}\ket{0}^{\otimes n}$}
\end{equation*}

\noindent We call the first circuit as the 'oracular circuit' and the second one as the 'oracle oblivious circuit'. The 'oracle oblivious circuit' is the extreme case where $U_\omega$ is completely ignored.\\ \\
Distance between output state of the two circuits is defined as follows
\begin{equation*}
    D_T = \sum_{\omega=0}^{N-1}\norm{\ket{\psi^\omega_T}-\ket{\psi_T}}^2
\end{equation*}

\begin{claim}
After $T$ iterations, $D_T \le 4T^2$ i.e., after a few queries to $U_w$, $D_T$ does not change by much.
\end{claim}
\begin{proof}
Proof by induction. Trivially,
\begin{equation*}
    T = 0 \implies D_T \le 0
\end{equation*}
$D_T \le 4T^2$ holds for case $T = 0$.\\ \\
Assume $D_T \le 4T^2$ holds true.\\ \\
Consider $D_{T+1}$,
\begin{equation*}
\begin{split}
    D_{T+1} & = \sum_\omega\norm{\ket{\psi^\omega_{T+1}} - \ket{\psi_{T+1}}}^2\\
    & = \sum_\omega\norm{U_{T+1}U_\omega\ket{\psi^\omega_T}-U_{T+1}\ket{\psi_T}}^2\\
    & = \sum_\omega\norm{U_{T+1}(U_\omega\ket{\psi^\omega_T}-\ket{\psi_T})}^2\\
\end{split}
\end{equation*}

\begin{equation*}
\begin{split}
    \norm{U(\ket{\psi}-\ket{\phi})}^2 & = [U(\ket{\psi}-\ket{\phi})(\ket{\psi}-\ket{\phi})^\dagger U^\dagger]^2\\
    & = (\ket{\psi}-\ket{\phi})(\ket{\psi}-\ket{\phi})^\dagger\\
\end{split}
\end{equation*}

\begin{equation*}
\begin{split}
    \therefore \sum_\omega\norm{U_{T+1}(U_\omega\ket{\psi^\omega_T}-\ket{\psi_T})}^2 & = \sum_\omega\norm{U_\omega\ket{\psi^\omega_T}-\ket{\psi_T}}^2\\
    & = \sum_\omega\norm{U_\omega(\ket{\psi^\omega_T} - \ket{\psi_T}) + (U_\omega - I)\ket{\psi_T}}^2
\end{split}
\end{equation*}

\begin{boxthm}[Cauchy schwarz inequality]
\label{thm:cauchy_schwarz}
\begin{equation*}
    |\langle u,v\rangle|^2 \le \langle u,u\rangle.\langle v,v\rangle
\end{equation*}
\begin{equation*}
    |\langle u,v\rangle| \le \norm{u}.\norm{v} \tag{$\norm{u} = \sqrt{\langle u,u \rangle}$}
\end{equation*}
\end{boxthm}

\begin{equation*}
    \therefore \norm{\ket{\psi^\omega_{T+1}}-\ket{\psi_{T+1}}}^2 \le \norm{\ket{\psi^\omega_T}-\ket{U_\omega-I}\ket{\psi_\omega}}^2 + \norm{(U_\omega - I)\ket{\psi_T}}^2 + 2\norm{\ket{\psi^\omega_T}-\ket{\psi_T}}\norm{(U_\omega-I)\ket{\psi_T}} \tag{Using cauchy schwarz}
\end{equation*}

W.k.t
\begin{equation*}
    \norm{(U_\omega)\ket{\psi_T}}^2 = \norm{-2\ket{\omega}\braket{\omega|\psi_T}}^2 = 4|\braket{\omega|\psi_T}|^2
\end{equation*}
\begin{equation*}
\begin{split}
    \therefore D_{T+1} & \le \sum_\omega\norm{\ket{\psi^\omega_T}-\ket{\omega_T}}^2 + 4\sum_\omega|\braket{\omega|\psi_T}|^2 + 4\sum_\omega\norm{\ket{\psi^\omega_T}-\ket{\psi_T}}.|\braket{\omega|\psi_T}|\\
    & \le D_T + 4 + 4\sum_\omega\norm{\ket{\psi^\omega_T}-\ket{\psi_T}}.|\braket{\omega|\psi_T}|
\end{split}
\end{equation*}

\begin{equation*}
    \sum_\omega\norm{\ket{\psi^\omega_T}-\ket{\psi_T}}.|\braket{\omega|\psi_T}| \le (\sum_\omega\norm{\ket{\psi^\omega_T}-\ket{\psi_T}}^2)^\frac{1}{2}(\sum_\omega|\braket{\omega|\psi_T}|^2)^\frac{1}{2} = \sqrt{D_T} \tag{Using cauchy schwarz}
\end{equation*}

\begin{equation*}
\begin{split}
    \therefore D_{T+1} & \le D^2_T + 4 + 4\sqrt{D_T}\\
    & \le 4T^2 + 8T + 4\\
    & \le 4(T+1)^2\\
\end{split}
\end{equation*}

\end{proof}

\begin{lemma}
\label{lemma:lemma1}
For any quantum state $\ket{\phi}$ and computational basis state $\ket{\omega}$ s.t. $\omega \epsilon \{0,1\}^n$
\begin{equation*}
    \sum_\omega\norm{\ket{\phi}-\ket{\omega}}^2 \ge 2N - 2\sqrt{N}
\end{equation*}
\end{lemma}

\begin{proof}
\begin{equation*}
\begin{split}
    \sum_\omega\norm{\ket{\phi}-\ket{\omega}}^2 & = \sum_\omega(|\braket{\phi|\phi}| + |\braket{\omega|\omega}| - \braket{\phi|\omega} - \braket{\omega|\phi})\\
    & = 2N - \sum_\omega\{\braket{\phi|\omega}+\braket{\omega|\phi}\}\\
    & \ge 2N - 2\sum_\omega|\braket{\phi|\omega}|
\end{split}
\end{equation*}

\begin{equation*}
    \sum_\omega|\braket{\phi|\omega}|.1 \le \sqrt{\sum_\omega|\braket{\phi|\omega}|^2}\sqrt{\sum_\omega1^2} \le \sqrt{N} \tag{Using cauchy schwarz}
\end{equation*}

\begin{equation*}
    \therefore \sum_\omega\norm{\ket{\phi}-\ket{\omega}}^2 \ge 2N - 2\sqrt{N}
\end{equation*}
\end{proof}

\begin{claim}
For algorithm using 'oracular circuit' to succeed w.p $\ge \frac{1}{2}$ ie., $|\braket{\omega|\psi^\omega_T}|^2 \ge \frac{1}{2} \forall \omega \epsilon \{0,1\}^n$ then $D_T = \Omega(N)$.
\end{claim}
\begin{proof}
\begin{equation*}
    D_T = \sum_\omega\norm{(\ket{\psi^\omega_T}-\ket{\omega})+(\ket{\omega}-\ket{\psi_T})}^2
\end{equation*}
\begin{equation*}
    D_T \ge \sum_\omega\norm{\ket{\psi^\omega_T}-\ket{\omega}}^2 + \sum_\omega\norm{\ket{\omega}-\ket{\psi_T}}^2 -2\sum_\omega\norm{\ket{\psi^\omega_T}-\ket{\omega}}.\norm{\ket{\omega}-\ket{\psi_T}} \tag{$\norm{a+b} \ge \norm{a-b}$}
\end{equation*}
\begin{equation*}
    \therefore D_T \ge (\sum_\omega\norm{\ket{\psi^\omega_T}-\ket{\omega}} - \sum_\omega\norm{\ket{\omega}-\ket{\psi_T}})^2
\end{equation*}
Using lemma \ref{lemma:lemma1}, we get
\begin{equation}
\label{eqn:claim2_1}
    \sum_\omega\norm{\ket{\omega}-\ket{\psi_T}}^2 \ge 2N - 2\sqrt{N}
\end{equation}
\begin{equation}
\label{eqn:claim2_2}
    \sum_\omega\norm{\ket{\psi^\omega_T}-\ket{\omega}}^2 = \sum_\omega(2 - 2Re(\braket{\omega|\psi^\omega_T})) \le (2 - \sqrt{2})N
\end{equation}

Using \ref{eqn:claim2_1} and \ref{eqn:claim2_2}, we get
\begin{equation*}
\begin{split}
    D_T & \ge [(2N-2\sqrt{N})^\frac{1}{2}-(2N-\sqrt{2}N)^\frac{1}{2}]^2\\
    & \ge [N^\frac{1}{2}\{(2-2\frac{2}{\sqrt{N}})^\frac{1}{2}-(2-\sqrt{2})^\frac{1}{2}\}]^2\\
    & \ge C.N\\
    D_T & = \Omega(N)\\
\end{split}
\end{equation*}

\end{proof}

\section{Amplitude estimation}
\noindent Goal is to increase the amplitude of certain states similar to what is done in "grover's search algorithm".

\subsection{Classical equivalent:}

\begin{claim}
Let algorithm $A$ output the desired solution w.p $p(<<1)$. To observe the desired solution, the expected number of times to run algorithm $A$ is $\frac{1}{p}$.
\end{claim}
\begin{proof}
\begin{equation*}
    Pr[\text{atleast 1 success in $k$ trials}] = 1 - (1-p)^k
\end{equation*}
\begin{equation*}
\begin{split}
    (1-p)^k & = \sum_{l=0}^{k}(-1)^l{k\choose l}p^l = \sum_{l=0}^{k}\frac{(-1)^lk!}{(k-l)!l!}p^l\\
    & = \sum_{l=0}^{k}(-1)^l\frac{(k-l+1)(k-l+2)\hdots k}{l!}p^l\\
    & \le \sum_{l=0}^{k}(-1)^l\frac{k^l}{l!}p^l = e^{-pk}\\
\end{split}
\end{equation*}

For $(1-p)^k \le e^{-pk} \le \epsilon$ to hold, $k \ge \frac{1}{p}\log(\frac{1}{\epsilon})$. Therefore, $1-(1-p)^k \ge 1-\epsilon \; \forall k \ge \frac{1}{p}\log(\frac{1}{\epsilon})$.
\end{proof}

\bibliographystyle{plainnat}
\bibliography{references}

\end{document}

