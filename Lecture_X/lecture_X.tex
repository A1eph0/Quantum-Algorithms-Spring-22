\documentclass[11.5pt, paper=a4]{article}

\usepackage[utf8]{inputenc}
\usepackage[english]{babel}
\usepackage[T1]{fontenc}

\usepackage{amsmath, amssymb, amscd, amsthm, amsfonts, mathtools}
\usepackage[left=2cm, right=2cm, top=1.5cm]{geometry}

\usepackage{graphicx}
\usepackage{hyperref}
\usepackage{physics}
\usepackage{tikz}
\usepackage{url}
\usepackage[square,numbers]{natbib} \usepackage{tabularx}

\usepackage{braket}
\usepackage{thmtools}
\usepackage{float}

%%% Theorem Style
\theoremstyle{definition}
\newtheorem{theorem}{Theorem}[section]
\newtheorem{definition}[theorem]{Definition}
\newtheorem{lemma}[theorem]{Lemma}
\newtheorem{conjecture}[theorem]{Conjecture}
\newtheorem{corollary}[theorem]{Corollary}

\numberwithin{theorem}{section}

%% Autoref prefixes
\renewcommand{\sectionautorefname}{Section}
\renewcommand{\subsectionautorefname}{Section}
\renewcommand{\subsubsectionautorefname}{Section}
\renewcommand{\figureautorefname}{Figure}
\def\theoremautorefname{Theorem}
\def\lemmaautorefname{Lemma}
\def\definitionautorefname{Definition}
\def\conjectureautorefname{Conjecture}
\def\algorithmautorefname{Algorithm}

%% Writing algorithms

\usepackage{algorithm} % captioning 
\usepackage{algpseudocode}

% \def\NoNumber#1{{\def\alglinenumber##1{}\State #1}\addtocounter{ALG@line}{-1}}

\graphicspath{{./Lecture_X/images/}}

\title{Quantum Algorithms, Spring 2022: Lecture X Scribe}

\author{Scribes' Names}

\date{\today}

\begin{document}

\maketitle

\section{Recap}

A short recap of the previous lecture. Scribe the current lecture from the next section. Remove the instructions section.

\section{Instructions}

\begin{enumerate}
\item Do not use all CAPS in titles. Replace the X with lecture number. Replace scribes' names with your names.

\item Use notation used in the class. (vectors, variables, matrices, sets, etc.)

\item All variables must be in math mode. That is \$ \$. 
For example, we write  $n$-dimensional and not n-dimensional. 

\item Center all tables and figures. 

\item Number and label all important equations and leave out the unimportant ones. 

\item You can use \begin{verbatim}
\autoref{eq:use_suitable_name}
\end{verbatim} to refer equations/tables/figures.

\item Use \begin{verbatim}
    \citet{example}
\end{verbatim} to refer to the bibliography. Example: \citet{example} is an example.

\item Do not use boldface to emphasize anything. Use command emph provided by latex.

\item As far as possible create your own examples/figures.
If any figure is downloaded from internet, please mention appropriate image credits.

\item Do not create images from textbook and paste them. You will get zero marks.

\item If you are new to latex, \href{https://en.wikibooks.org/wiki/LaTeX/Mathematics}{this hyperlink} might be useful. Overleaf has good resources too.

\item Submit a zip file that contains all the required resources, including the PDF of your scribe, source, images, bibliography, etc.

\item Please use sensible conventions, indentations and avoid annoying mistakes in the course file. 

\end{enumerate}

Example on how to use theorem, definition, lemma, etc. environments:

\begin{verbatim}
    \begin{theorem}[Theorem Name]
    \label{thm:inappropriate_text}

        Theorem text

    \end{theorem}

    \begin{proof}

    Proof by Obviousness. Hence proved. 
    
    \end{proof}


    \begin{corollary}[Corollary Name]
    \label{cor:less_inappropriate_text}
        Corollary text
    \end{corollary}
    
    \begin{proof}
        Proof by Lack of sufficient time. Hence proved.
    \end{proof}

\end{verbatim}

You may play around to find out more.

\bibliographystyle{plainnat}
\bibliography{references}

\end{document}

