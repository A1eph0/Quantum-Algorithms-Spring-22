\documentclass[11.5pt, paper=a4]{article}

\usepackage[utf8]{inputenc}
\usepackage[english]{babel}
\usepackage[T1]{fontenc}

\usepackage{amsmath, amssymb, amscd, amsthm, amsfonts, mathtools}
\usepackage[left=2cm, right=2cm, top=1.5cm]{geometry}

\usepackage{graphicx}
\usepackage{hyperref}
\usepackage{physics}
\usepackage{tikz}
\usetikzlibrary{quantikz}
\usepackage{url}
\usepackage[square,numbers]{natbib}
\usepackage{tabularx}

\usepackage{braket}
\usepackage{thmtools}
\usepackage{float}

%%% Theorem Style
\theoremstyle{definition}
\newtheorem{theorem}{Theorem}[section]
\newtheorem{definition}[theorem]{Definition}
\newtheorem{lemma}[theorem]{Lemma}
\newtheorem{conjecture}[theorem]{Conjecture}
\newtheorem{corollary}[theorem]{Corollary}

\numberwithin{theorem}{section}

%% Autoref prefixes
\renewcommand{\sectionautorefname}{Section}
\renewcommand{\subsectionautorefname}{Section}
\renewcommand{\subsubsectionautorefname}{Section}
\renewcommand{\figureautorefname}{Figure}
\def\theoremautorefname{Theorem}
\def\lemmaautorefname{Lemma}
\def\definitionautorefname{Definition}
\def\conjectureautorefname{Conjecture}
\def\algorithmautorefname{Algorithm}

%% Writing algorithms

\usepackage{algorithm} % captioning 
\usepackage{algpseudocode}

% \def\NoNumber#1{{\def\alglinenumber##1{}\State #1}\addtocounter{ALG@line}{-1}}

\graphicspath{{./Lecture_8/images/}}

\title{Quantum Algorithms, Spring 2022: Lecture 8 Scribe}

\author{Pahulpreet Singh \and Shreyas Pradhan}

\date{February 04, 2022}

\begin{document}

\maketitle

\section{Recap}

\subsection{Simon's Algorithm}

Given a blackbox $U_f$, for a function $f$ from $n$ bit-strings to $n$ bit-strings i.e $f:\{0,1\}^n\rightarrow \{0,1\}^n$, with the promise that $f$ is $2-1$ function defined as follows:$\forall$ input $(x,y)\in \{0,1\}^n, f(x) = f(y), \iff x=y\oplus s$. \\ [2mm]
Here, $s\in (0,1)^n$ is secret string. How many queries to $U_f$ are needed to find $s$?

\subsubsection{Classical Algorithm}

We proved that the complexity is $O(\sqrt{2^n})$. \big[ Matching lower bound exists. \big]

\subsubsection{Quantum Algorithm}

\begin{figure}[hb]
    \centering
    \begin{quantikz}
        \lstick{$\ket {0}^{\otimes n}$} & \gate[wires=1][1cm]{H^{\otimes n}} & \gate[wires=2][2cm]{U_f} & \qw & \gate[wires=1][1cm]{H^{\otimes n}} & \meter{} & \cw\\
        \lstick{$\ket {0}^{\otimes n}$} & \qw & \qw & \qw & \qw & \meter{} & \cw
    \end{quantikz}
    \caption{Quantum Implementation of Simon's Algorithm} \label{fig:1}
\end{figure}

After running the circuit operations we have:
\begin{itemize}
    \item When $x \oplus y = 0^n $ (i.e. $s=0^n$), the measurement results in each string $y \in \{ 0, 1\}^n$ with probability $p_y = \frac{1}{ 2^n }$
    \item in the cas\ $x \oplus y = s $ (where $s \neq 0^n$), the probability to obtain each string $y \in \{ 0, 1\}^n$ is given by
          $$p_y =
              \begin{cases}
                  {1/2^{n-1}} & \quad \text{if } y \cdot s \text{ is even} \\
                  0           & \quad \text{if } y \cdot s \text{ is odd}  \\
              \end{cases}$$
\end{itemize}
Thus, in both cases, the measurement results is some string $y \in \{ 0, 1\}^n$ that satisfies $s \cdot y = 0$, and the distribution is uniform over all of the strings that satisfy this constraint and this is enough information to determine $s$.\\[2mm]
We repeat the above process $n - 1$ times to get $n - 1$ strings $y_1, y_2, \dots, y_{n-1} \in \{ 0, 1\}^n$, such that
$$
    \begin{cases}
        y_1 \cdot s     & = 0    \\
        y_2 \cdot s     & = 0    \\
                        & \vdots \\
        y_{n-1} \cdot s & = 0
    \end{cases}
$$
and we can use this set of linear \
equations to find $s$. We only get a unique non-zero solution $s$ if $y_1, y_2,
    \dots, y_{n-1} \in \{ 0, 1\}^n$ are linearly independent (P $\geq 1/4$).

\begin{itemize}
    \item Use Gaussian elimination to obtain $s$. \big[ Needs $O(n^3)$ time, but no additional queries to $U_f$ \big]
    \item Check if $f(0\cdots 0) = f(s)$. If yes, we are done. If not, repeat this whole procedure a few times.
    \item Repeating the process $O(1)$ times, i.e. $O(n)$ number of queries to $U_f$, are enough to guarantee a high success probability.

          For $k$ runs, \hspace{50px} $P(\text{success}) = 1 - \big( \displaystyle \frac{3}{4} \big)^k = 1 - \epsilon$
          \hspace{50px} where $k \approxeq \displaystyle \frac{log(1/\epsilon)}{log(4/3)}$
    \item Query complexity: $O(n)$ \hspace{100px} Quantum lower bound: $\Omega(n)$
\end{itemize}

For an exact version of Simon's Algorithm, refer to \citet{1}.

\section{Q.F.T. and its applications}

\subsection{Elementary Concepts}

\subsubsection{Complex Roots of Unity}

\begin{align*}
    \omega^N = 1  \hspace{100px}                  & \omega = e^{{2\pi\iota}/{N}} \\
    1 + \omega + \omega^2 + \cdots + \omega^{N-1} & = 0
\end{align*}



\begin{figure}[ht]
    \def\n{5}
    \centering
    \begin{tikzpicture}[
            scale = 1.5,
            dot/.style={draw,fill,circle,inner sep=1pt}
        ]
        \draw[->] (-2,0) -- (2,0) node[below] {$\mathbb{R}$};
        \draw[->] (0,-2) -- (0,2) node[left] {$\mathbb{I}$};
        \draw[help lines] (0,0) circle (1);

        \node[dot,label={below right:$O$}] (O) at (0,0) {};
        \foreach \i in {1,...,\n} {
                \node[dot,label={\i*360/\n-(\i==\n)*45:$\omega^{\i}$}] (w\i) at (\i*360/\n:1) {};
                \draw[->] (O) -- (w\i);
            }
        \draw[->] (0:.3) arc (0:360/\n:.3);
        \node at (360/\n/2:.5) {$\alpha$};
    \end{tikzpicture}
    \caption{Roots of Unity for N=5} \label{fig:2}
\end{figure}


\subsubsection{Discrete Fourier Transform}

A complex vector $\big(x_0, x_1, \cdots, x_{N-1} \big)$ is transformed to another complex vector
$\big( y_0, y_1, \cdots, y_{N-1} \big)$ such that:

\begin{equation}
    y_k = \sum_{j=0}^{N-1} x_j \cdot \omega^{jk}
\end{equation}

\subsection{Quantum Fourier Transform}

For Q.F.T. (Quantum Fourier Transform), the same transformation is carried out, except that it maps
quantum states to quantum states.

\paragraph{Example}
Say $N = 2^n$ and $\{ \ket{0}, \ket{1}, \cdots, \ket{N-1} \}$ are computational basis states. Then
QFT on a state $\ket{j}$ is defined as:

\[
    \ket{j} \xmapsto{QFT} \displaystyle \frac{1}{\sqrt{N}} \sum_{k=0}^{N-1} \omega^{jk} \ket{k}
\]

where $\displaystyle \frac{1}{\sqrt{N}}$ is the normalization factor. Generalizing this to an ensemble of states,

\begin{align}
    \sum_{j=0}^{N-1} x_j\ket{j} \xmapsto{QFT} & \sum_{k=0}^{N-1} y_k \ket{k}                   \\
                                              & \big[ \text{where } y_k = \sum_{j=0}^{N-1} x_j
        \frac{\omega^{jk}}{\sqrt{N}} \big] \nonumber
\end{align}


We define $F_N$ as the unitary operator for QFT modulo $N$.

\[
    F_N\ket{j} \xmapsto{} \displaystyle \frac{1}{\sqrt{N}} \sum_{k=0}^{N-1} \omega^{jk} \ket{k}
\]

And the $(j, k)^{th}$ entry of the matrix is given by:

\[
    (F_N)_{j,k} = \braket{k|F_N|j} = \displaystyle \frac{\omega^{jk}}{\sqrt{N}}
\]

From this, we can compute $F_N$ as:

\[
    F_N = \displaystyle \frac{1}{\sqrt{N}} \begin{pmatrix}
        1      & 1            & 1               & \cdots & 1                \\
        1      & \omega       & \omega^2        & \cdots & \omega^{N-1}     \\
        1      & \omega^2     & \omega^4        & \cdots & \omega^{2(N-1)}  \\
        \vdots &              &                 & \ddots & \vdots           \\
        1      & \omega^{N-1} & \omega^{2(N-1)} & \cdots & \omega^{(N-1)^2} \\
    \end{pmatrix}
\]

\begin{enumerate}
    \item Classically, for D.F.T., we need $O(N \log{N})$ steps / operations to F.F.T. (Fast Fourier Transform).
    \item For QFT, we shall require $O(\log^2{N})$ elementary gates.
    \item Instead of a vector, QFT outputs a quantum state!
    \item Classically, we have access to all the $y_k$ in the vector. This is not the case for QFT.
    \item At the end of QFT, we need to make a measurement. We observe some $\ket{k}$ with the probability $|y_k|^2$.
    \item When we sample from the output state, we do so according to the Fourier Transform coefficients. (This is called Fourier Sampling)
    \item So, we need to make clever use of QFT to harness its power for algorithmic applications!
\end{enumerate}




\bibliographystyle{plainnat}
\bibliography{references}

\end{document}

