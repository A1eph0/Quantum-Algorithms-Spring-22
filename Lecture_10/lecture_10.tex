\documentclass[11.5pt, paper=a4]{article}
\usepackage[utf8]{inputenc}
\usepackage[english]{babel}
\usepackage[T1]{fontenc}
\usepackage{amsmath, amssymb, amscd, amsthm, amsfonts, mathtools}
\usepackage[left=2cm, right=2cm, top=1.5cm]{geometry}
\usepackage{graphicx}
\usepackage{hyperref}
\usepackage{physics}
\usepackage{tikz}
\usepackage{url}
\usepackage[square,numbers]{natbib} \usepackage{tabularx}
\usetikzlibrary{quantikz}
\usepackage{braket}
\usepackage{thmtools}
\usepackage{float}
\theoremstyle{definition}
\newtheorem{theorem}{Theorem}[section]
\newtheorem{claim}{Claim}[section]
\newtheorem{definition}[theorem]{Definition}
\newtheorem{lemma}[theorem]{Lemma}
\newtheorem{conjecture}[theorem]{Conjecture}
\newtheorem{corollary}[theorem]{Corollary}
\numberwithin{theorem}{section}
\renewcommand{\sectionautorefname}{Section}
\renewcommand{\subsectionautorefname}{Section}
\renewcommand{\subsubsectionautorefname}{Section}
\renewcommand{\figureautorefname}{Figure}
\def\theoremautorefname{Theorem}
\def\lemmaautorefname{Lemma}
\def\definitionautorefname{Definition}
\def\conjectureautorefname{Conjecture}
\def\algorithmautorefname{Algorithm}
\usepackage{algorithm}
\usepackage{algpseudocode}
\usepackage{braket}
\usepackage{physics}
\usepackage{mathtools}
\usepackage{amsmath}
\usepackage[utf8]{inputenc}
\title{Quantum Algorithms, Spring 2022: Lecture 10 Scribe}
\author{Sravani Yanamandra, Srikar Kale}
\date{\today}
\begin{document}
\maketitle
\section{Recap}
\subsection{Properties of QFT:}
\begin{enumerate}
  \item \textbf{Invariance under shift:}
    Suppose,
    \begin{equation*}
        \sum\limits_{j} \alpha_j  \ket{j} \underrightarrow{QFT} \sum\limits_{k} \beta_k \ket{k} 
    \end{equation*}
    then, 
    \begin{equation*}
        \sum\limits_{j} \alpha_j  \ket{j+s} \underrightarrow{QFT} \sum\limits_{k} \omega^{sk} \beta_k \ket{k} 
    \end{equation*}
    \item \textbf{QFT maps a periodic superposition to a period superposition:}
  
    Let, 
    \begin{equation*}
        \ket{\psi} = \frac{1}{\sqrt{A}} \sum\limits_{k = 0}^{A-1} \ket{kr}
    \end{equation*}
    such that
    \begin{equation*}
      A=\frac{N}{r} \;,\; r | N
    \end{equation*} then,
    \begin{equation*}
      QFT\ket{\psi} = \frac{1}{\sqrt{r}} \sum\limits_{j = 0}^{r - 1} \ket{\frac{jN}{r}}
    \end{equation*}
    \end{enumerate}

\subsection{Quantum period finding:}
\begin{enumerate}
    \item \textbf{Problem statement:} 
    
    Given a blackbox $U_f$ for some boolean function $f:\{0,1\}^n \rightarrow \{0,1\}^m$ such that $f$ is a periodic function with $r << \sqrt{N}$ where $N = 2^n$ i.e, $f(x) = f(y) \Leftrightarrow  y =x(mod$ r$)$. How many queries to $U_f$ are required to find $r$?
    
    \item \textbf{Circuit:}
    \begin{center}
    \begin{quantikz}
        \lstick{$\ket{0}^{\otimes n}$} & \qwbundle{n} & \gate{H} & \gate[wires=2][2cm]{U_f} & \gate{QFT} & \meter{} \\
        & \lstick{$\ket{0}^{\otimes n}$} & \qwbundle{n} & & \meter{(*)}
    \end{quantikz}
    \end{center}

    \item \textbf{Equations:}
   
    Hadamard gate applied to first register:
    \begin{equation*}
        \ket{0}^{\otimes n}\ket{0}^{\otimes n} \xrightarrow{H^{\otimes n} \otimes I} \frac{1}{\sqrt{N}} \sum\limits_{x \in \{0,1\}^n}\ket{x}\ket{0}
    \end{equation*}
    
    $U_f$ applied to both registers
    \begin{equation*}
        \frac{1}{\sqrt{N}} \sum\limits_{x \in \{0,1\}^n}\ket{x}\ket{0} \xrightarrow{U_f} \frac{1}{\sqrt{N}} \sum\limits_{x \in \{0,1\}^n}\ket{x}\ket{f(x)}
    \end{equation*}
    
    Measuring 2nd register and observe $f(x_0)$: 
    \begin{equation*}
        \frac{1}{\sqrt{A}} \sum\limits_{k = 0}^{A-1}\ket{x_0 + kr}
    \end{equation*}
    
    Using the shift in variance property discussed in $Section \; 1.1$ (property 1)
    \begin{equation*}
        \frac{1}{\sqrt{A}} \sum\limits_{k = 0}^{A-1}\ket{x_0 + kr}  = \frac{1}{\sqrt{A}} \sum\limits_{k = 0}^{A-1}\ket{kr}    
    \end{equation*}
    
    Applying QFT on the remaining qubits
    \begin{equation*}
        \frac{1}{\sqrt{A}} \sum\limits_{k = 0}^{A-1}\ket{kr}  \xrightarrow{QFT}  \frac{1}{\sqrt{AN}} \sum\limits_{l = 0}^{N-1}\sum\limits_{k = 0}^{A-1} (\omega^{rl})^k\ket{l} 
    \end{equation*}
    
    \item \textbf{Amplitude of $\ket{l}$:}
    
    \begin{equation*}
        \alpha_l = \frac{1}{\sqrt{AN}} \sum\limits_{k = 0}^{A-1} (\omega^{rl})^k
    \end{equation*}
    
    \begin{equation*}
        \alpha_l = \begin{cases} 
        {\sqrt{\frac{A}{N}}} &\quad \text{if} \; \omega^{rl} = 1\\
         \frac{1}{\sqrt{NA}}\frac{(1-\omega^{rlA})}{(1-\omega^{rl})} &\quad \text{if} \; \omega^{rl} \neq 1\\
        \end{cases}
    \end{equation*}
    
    \item \textbf{We have 2 cases:} 
    \begin{enumerate}
        \item Case 1:
        \begin{equation*}
            r|N \implies A = \frac{N}{r}
        \end{equation*}
        \item Case 2:
        \begin{equation*}
            r \nmid N \implies A = \left\lceil \frac{N}{r}  \right\rceil \;or \left\lfloor \frac{N}{r} \right\rfloor  \; ,  \; A-1 \leq \frac{N}{r} \leq A+1
        \end{equation*}
    \end{enumerate}
    \item \textbf{We had already discussed Case 1 in the previous class and analysed it as follows:}
        \begin{equation*}
            r|N \implies A = \frac{N}{r}
        \end{equation*}
        \begin{equation*}
            \sqrt{\frac{r}{N}} \sum \limits_{k = 0}^{\frac{N}{r} - 1} \ket{kr} \rightarrow \frac{1}{\sqrt{r}} \sum \limits_{j = 0}^{r - 1} \ket{\frac{jN}{r}}
        \end{equation*}
        \begin{enumerate}
            \item Measuring the first register gives $s_1 \frac{N}{r}$
            \item Repeating $k = \theta(1)$ times we have $\{s_1\frac{N}{r}, s_2 \frac{N}{r}, s_3 \frac{N}{r}, ...., s_k \frac{N}{r} \}$
            \item With very high probability, $S_i$'s are relatively co prime
            \item gcd$\{s_1\frac{N}{r}, s_2 \frac{N}{r}, s_3 \frac{N}{r}, ...., s_k \frac{N}{r} \} =  \frac{N}{r} $ which gives us $r$ as $N$ is known.
            \item The additional cost here is $O(logN)$ from Euclid's GCD algorithm
            \item If the $S_i$'s are not relatively co prime, we wont be able to detect the mistake in the next step. To overcome this we can take $k$ to be large enough such that the probability is extremely high or repeat the algo and verify $r$ in the subsequent iteration.
            \item These are not deterministic algorithm's
        \end{enumerate}
\end{enumerate}
\section{Things to know for this class:}
\subsection{Properties of sin(x):}

\begin{enumerate}
    \item For $x \in \mathbb{R}$, $\theta < \frac{|sin(x)|}{|x|} < 1 \implies sin^2(x) < x^2$,  this can be derived using the Mean Value Theorem 
    \item For $0 \leq |x| \leq \frac{\pi}{2},  |sin(x)| \geq \frac{2|x|}{\pi}$
    
    This is because $\frac{|sin(x)|}{|x|}$ is a decreasing function for the interval $0 \leq |x| \leq \frac{\pi}{2}$. Let $ g(|x|) =\frac{|sin(x)|}{|x|} \; then \; g(|x|) \geq g(\frac{\pi}{2}) $ (for decreasing functions only)
    
    \item $sin^2(\pi q \pm \theta) = sin^2(\theta) \; as \;  sin(\pi q \pm \theta) = \pm sin(\theta) \; when \; q \in \mathbb{N}_0$
\end{enumerate}

\subsection{Continued fractions}
The idea of the continued fractions method is to describe real numbers in terms of integers alone.  A finite simple continued fraction is defined by a finite collection $a_0 , . . . , a_N$ of positive integers as
\begin{equation*}
    [a_0, a_1,....,a_N] = a_0 + \frac{1}{a_1 + \frac{1}{a_2 + \frac{1}{...+\frac{1}{a_N}}}}
\end{equation*}
We define the nth convergent $(0 \leq n \leq N )$ to this continued fraction to be $[a_0 , . . . , a_n ]$.
\\
Suppose we are trying to decompose 31/13 as a continued fraction. 

The first step of the continued fractions algorithm is to \textbf{split} 31/13 into its integer and fractional part,
\begin{equation*}
    \frac{31}{13} = 2 + \frac{5}{13}
\end{equation*}

Next we \textbf{invert} the fractional part, obtaining
\begin{equation*}
    \frac{31}{13} = 2 + \frac{1}{\frac{13}{5}}
\end{equation*}

And we keep on going forward with the \textbf{split and invert} procedure, we would end up with

\begin{equation*}
      \frac{31}{13} = 2 + \frac{1}{2 +\frac{1}{1+ \frac{1}{1+\frac{1}{2}}}}
\end{equation*}
\\
\begin{enumerate}
    \item It’s clear that the continued fractions algorithm terminates after a finite number of ‘split and invert’ steps for any rational number, since the numerators which appear are strictly decreasing.
    \item The continued fractions algorithm provides an unambiguous method for obtaining a continued fraction expansion of a given rational number.
    \item The only possible ambiguity comes at the final stage, because it is possible to split an integer in two ways, either $a_N = a_N$ , or as $a_N = (a_N - 1) + 1/1 $, giving two alternate continued fraction expansions.
    \item This ambiguity is actually useful, since it allows us to assume without loss of generality that the continued fraction expansion of a given rational number has either an odd or even number of convergent, as desired.
    \item How quickly does this termination occur?
        It follows that if $x = \frac{p}{q}$ is a rational number, $p \; \& \; q$ are $L$ bit integers, then the continued fraction expansion for $x$ can be computed using $O(L^3)$ operations – $O(L)$ ‘split and invert’ steps, each using $O(L^2)$ gates for elementary arithmetic.
\end{enumerate}

\subsection{Useful claim 1:}
\begin{claim}
2 distinct rational numbers with denominators $\leq r$ are atleast $\frac{1}{r^2}$ apart
\end{claim}
\begin{proof}
Let $Z = \frac{X}{Y}$ and $Z' = \frac{X'}{Y'}$ with $(Y,Y')\leq r$ then 
\begin{equation*}
    |Z - Z'| = |\frac{XY'-X'Y}{YY'}|
\end{equation*}
The numerator will be $\geq 1$ while the denominator will be $\leq r^2$ so we can say 
\begin{equation*}
    |Z - Z'| \geq \frac{1}{r^2}
\end{equation*}
\end{proof}
\subsection{Useful claim 2}
\begin{claim}
Let $p,q \in Z^{+}$, and $x \in Q$ that satisfy the following: 
\begin{equation*}
    |x - \frac{p}{q}| \leq \frac{1}{2q^2}
\end{equation*}
Also let $c = (c_0, c_1,....,c_m)$ be the continued fraction expansion of $x$ then there exists some $Q \in \{0,1,2,...M\}$ such that the $Q^{th}$ convergent $C_Q = (c_0, c_1,....,c_Q)$ is exactly $\frac{p}{q}$. Further more choosing $M = O(log N)$ suffices, where $p \;\&\; q$ are $logN$ bit integers.
\end{claim}
\begin{proof}
The proof of this claim is available in the Quantum Computation and Quantum Information textbook by Michael A. Nielsen $\&$ Isaac L. Chuang, Appendix: Number Theory.
\end{proof}

\section{Main crux of lecture 10:}
\subsection{Quantum period finding, Case 2:}
    We know,
    \begin{equation*}
        r \nmid N \implies A = \left\lceil \frac{N}{r}  \right\rceil \;or \left\lfloor \frac{N}{r} \right\rfloor  \; ,  \; A-1 \leq \frac{N}{r} \leq A+1
    \end{equation*}
    Let 
    \begin{equation*}
        \alpha_l = \frac{1}{\sqrt{NA}}\frac{(1-\omega^{rlA})}{(1-\omega^{rl})}, \omega = e^{\frac{i2\pi}{N}}
    \end{equation*}
    In this case, we will argue that $l$ is still very close to some $\frac{jN}{r}$
    \begin{equation*}
        |\alpha_l|^2 = \frac{1}{NA} \frac{|1- e^{\frac{2irlA\pi}{N}}|^2}{|1-e^{\frac{2irl\pi}{N}}|^2}
    \end{equation*}
    We know that
    \begin{equation*}
        |1-e^{i\theta}| = |\frac{2ie^{\frac{i\theta}{2}} (e^{\frac{-i\theta}{2}} - e^{\frac{i\theta}{2}})}{2i}|
    \end{equation*}
    \begin{equation*}
        |1-e^{i\theta}|  = 2|-i e^{\frac{i\theta}{2`}}sin(\frac{\theta}{2})|
    \end{equation*}
    Which implies,
    \begin{equation*}
        |1-e^{i\theta}|^2 = 4sin^2(\frac{\theta}{2})
    \end{equation*}
    Coming back to $\alpha_l$
    \begin{equation*}
        |\alpha_l|^2 = \frac{1}{NA} \frac{sin^2(\frac{rlA\pi}{N})}{sin^2(\frac{rl\pi}{N})} 
    \end{equation*}
    So the probability of observing some $l$ is given by $|\alpha_l|^2$ which is $\frac{1}{NA} \frac{sin^2(\frac{rlA\pi}{N})}{sin^2(\frac{rl\pi}{N})} $. 
    \\
    What is the probability of $l$ being close to some multiple of $\frac{N}{r}$?
    \\
    Consider, $l = \frac{mN}{r} + \delta_m$ where $m \in \{0,1,..., r-1 \}$,  $\delta_m$ is a very small value. We know that $|\delta_m| < 0.5$ (we want this kind of accuracy, rather). The intuition behind this is that the nearest integer from $\frac{mN}{r}$ would be less than $0.5$ distance away.
    \\
    We now plug in the value of $l$ 
    \begin{equation*}
        \frac{rlA\pi}{N} = \frac{\pi r A}{N}(\frac{mN}{r}+ \delta_m)
    \end{equation*}
    \begin{equation*}
        \frac{rlA\pi}{N} = \pi mA + \frac{\pi r A \delta_m}{N}
    \end{equation*}
    Similarly
    \begin{equation*}
        \frac{rl\pi}{N} = \pi m + \frac{\pi r  \delta_m}{N}
    \end{equation*}
    \\ 
    Using the property 3 from $Section \; 2.1$ and plugging the above information into $|\alpha_l|^2$,
     \begin{equation*}
        |\alpha_l|^2 = \frac{1}{NA} \frac{sin^2(\frac{rlA\pi}{N})}{sin^2(\frac{rl\pi}{N})} 
    \end{equation*}
    \begin{equation*}
        |\alpha_l|^2 = \frac{1}{NA} \frac{sin^2(\frac{\pi r A \delta_m}{N})}{sin^2(\frac{r\pi \delta_m}{N})} 
    \end{equation*}
    \\
    So to get a lower bound on $|\alpha_l|^2$ we use property 2 to the numerator and property 1 to the denominator $present \; in \; Section \; 2.1$. We get
    \begin{equation*}
        |\alpha_l|^2 \geq \frac{4A}{\pi^2 N}
    \end{equation*}
    \begin{equation*}
        |\alpha_l|^2 \geq const. \frac{A}{N}  
    \end{equation*}
    \\
    To be able to apply the property for the numerator we need to show $0 \leq |\frac{rlA\pi}{N}| \leq \frac{\pi}{2}$, we know that
    \begin{equation*}
        A-1 \leq \frac{N}{r} \leq A+1
    \end{equation*}
    \begin{equation*}
        \frac{N}{r} - 1 \leq A \leq \frac{N}{r} + 1
    \end{equation*}
    \begin{equation*}
        1- \frac{r}{N} \leq \frac{Ar}{N} \leq 1+ \frac{r}{N}
    \end{equation*}
    \\
    By assumption  $r << \sqrt{N}$, $\frac{rA}{N}\simeq 1$, $\frac{r}{N} \simeq o(\frac{1}{\sqrt{N}})$
    \\
    So,
    \begin{equation*}
        \frac{\pi r \delta_m A}{N} \simeq \pi \delta_m
    \end{equation*}
    \begin{equation*}
        0 \leq \frac{\pi r \delta_m A}{N} \leq \frac{\pi}{2} \pm O(\frac{r}{N})
    \end{equation*}
    We can ignore the right most term and apply this property to the numerator:
    \begin{equation*}
        |\alpha_l|^2 \geq \frac{4A}{\pi^2 N}
    \end{equation*}
    \begin{equation*}
        |\alpha_l|^2 \geq \frac{4}{\pi^2 r}
    \end{equation*}
    \\    
    With a probability $\geq \frac{4}{\pi^2 r} $ we observe some $l$ such that $ l = \frac{mN}{r} + \delta_m $
    \begin{equation*}
        |l - \frac{mN}{r}| = \delta_m
    \end{equation*}

    \begin{equation*}
        \implies |l - \frac{mN}{r}| < 0.5 \; (as \; \delta_m < 0.5) for \; m \in \{0,1,2,....,r-1 \} 
    \end{equation*}
    \\
    Therefore, the probability of observing any such $'m'$
    \begin{equation*}
        \geq  \frac{4}{\pi^2 r} * r 
    \end{equation*}
    \begin{equation*}
        \geq  \frac{4}{\pi^2}
    \end{equation*}
    \begin{equation*}
        \geq 0.4
    \end{equation*}
\\
\textbf{For case 1: } $r|N$, we observed $\frac{mN}{r}$ exactly. If we observe some $'bad'$ value of $l$ we still follow the same procedure and get some $'r'$ and if its erroneous, we repeat it again.
\\
Over here, we get some $'l'$ close to $\frac{mN}{r}$ but we dont know which $\frac{mN}{r}$ it is. We need to extract this closeness multiple to $\frac{N}{r}$. Lets say we do that erroneously. We do the procedure a few times and it fails one of the time, and we dont get the correct value of $'r'$ but we are guaranteed that it will succeed with probability $\geq 0.4$
\\
\\
\textbf{So far:}
\begin{enumerate}
    \item We have observed some $'l'$ such that with a probability $\geq \frac{4}{\pi ^ 2}$, $|l - \frac{mN}{r}| < 0.5$
    \item From the above equation $|\frac{l}{N} - \frac{m}{r}| < \frac{1}{2N} < \frac{1}{r^2} \; as \; (r<< \sqrt{N}) $
    \item $\frac{l}{N} \;\&\; \frac{m}{r}$ are 2 rational numbers
    \item we know $l \;\&\; N$, we dont know $m \;or\; r$ 
\end{enumerate}

We need information about $\frac{m}{r}$. From the claim in $section \; 2.3$ we can say that $\frac{m}{r}$ is the only fraction with denominator $\leq r$ that is within $\frac{1}{2r^2}$ of $\frac{l}{N}$. The reason being if any other fraction existed then that distance would be at least $\frac{1}{r^2}$ away from $\frac{m}{r}$.
\\
Now there exists a classical procedure that allows us to exactly obtain $\frac{m}{r}$, starting from $\frac{l}{N}$ provided $|\frac{l}{N} - \frac{m}{r}| < \frac{1}{2r^2}$. This procedure is known as continued fraction expansion. This has been explained to some extent in $Section 2.2$.
\\
Using the claim in $Section \; 2.4$ Starting from $\frac{l}{N}$, we can exactly obtain $\frac{m}{r}$ in $O(log N)$ steps. After the continued fraction algorithm, we end up with $\frac{m}{r}$.
\\
Repeat QFT algorithm and the continued fraction algorithm k times such that we get:
\begin{equation*}
    \{ \frac{m_1}{r},  \frac{m_2}{r}, ...,  \frac{m_k}{r}\}
\end{equation*}
\\
then,
\begin{equation*}
    gcd\{ \frac{m_1}{r},  \frac{m_2}{r}, ...,  \frac{m_k}{r}\} \; to \; get \; r
\end{equation*}
\\
where $m_1,m_2, ... , m_k$ are relatively coprime with very high probability. The GCD calculation will take $O(log N)$ steps.

\subsection{Analysis of QFT period finding, case 2:}
\begin{enumerate}
    \item Query complexity: $K = \theta(1)$ to $U_f$
    \item Additional costs:
    \begin{enumerate}
        \item $O(log N)$ overhead for GCD and continued fractions calculations
        \item $log N$ number of Hadamard gates needed
        \item QFT circuit: $O(log^2 N)$ elementary gates.
    \end{enumerate}
\end{enumerate}

\end{document}
